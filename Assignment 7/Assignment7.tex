\documentclass[a4paper]{article}

\usepackage[english]{babel}
\usepackage[utf8]{inputenc}
\usepackage{amsmath}
\usepackage{graphicx}
\usepackage[colorinlistoftodos]{todonotes}
\usepackage[export]{adjustbox}[2011/08/13]
\usepackage{float}
\usepackage[T1]{fontenc}
\usepackage{bm}

\title{Behaviour Dynamics in Social Networks - Assignment 7}

\author{Maria Hotoiu, Federico Tavella}

\date{\today}

\begin{document}
\maketitle

\begin{abstract}
Verification by mathematical analysis of stationary points.
\end{abstract}

\section{Determine equilibria for a constant stimulus}

The values chosen for the parameters are: repetition=30, duration=30, $\eta_{1}$=0.4, $\eta_{2}$=0.3, $\mu_{1}$=0.8 and $\mu_{2}$=0.9.

\subsection{Question 1}
The final equilibrium values for the states rep, prep and feel based on observations in the simulation are:\\
\textbf{rep}: 1 \\
\textbf{prep}: 0.991535551	\\
\textbf{feel}: 0.89068188

\subsection{Question 2}

The observed equilibrium values based on observation in the simulation:\\
$\omega_{1}$=0.832149364\\
$\omega_{2}$=0.898285371\\

\noindent 	The predicted equilibrium values based on mathematical analysis: \\
$\omega_{1}$=0.832149364\\
$\omega_{2}$=0.898285371\\

The two sets of values for the connection weights are equal, therefore the accuracy is 0.

\subsection{Question 3}

The differences between the aggregated impact and the values for the two adaptive connections for these equilibria vary from 0 to 0.2 in the first case (aggimpact-$\omega_{1}$) and from 0 to 0.1 in the second case (aggimpact-$\omega_{2}$). 
By exploring the difference between the aggregated impact and $\omega_{1}$ and $\omega_{2}$ in the stationary points we can observe that the accuracy is 0<$10^{-2}$. 


\subsection{Question 4}


$c_{feel}$($\omega_{2}$prep(t))= feel(t) $\implies$ \\ \\
id($\omega_{2}$prep(t)) = feel(t) $\implies$ \\ \\
id($\omega_{2}$\underline{prep}) = \underline{feel} $\implies$\\ \\
$\omega_{2}$\underline{prep} = \underline{feel}\\ \\

If we replace the values with the observed ones we get: 0.898285371*0.991535551=0,890681512430035. This means that the accuracy is 0,0000004<$10^{-2}$, so we can say that the equation is verified.


\section{Determine stationary points for an alternating stimulus}

Formula used for accuracy: (observed value–predicted value)/predicted value=error. In our case, the observed values are $\omega_{1}$ and $\omega_{2}$ and the predicted value is the aggregated impact.

\subsection{Question 1}

The values chosen for the parameters are: repetition=40, duration=20, $\eta_{1}$=0.4, $\eta_{2}$=0.3, $\mu_{1}$=0.8 and $\mu_{2}$=0.9.

Examples of stationary point:. \\
\textbf{rep}: t0-t18, t20-t38 etc. \\
\textbf{prep}: t77, t78, t79, t115, t116 etc.	\\
\textbf{feel}: t116, t117, t118 etc.

Examples of equilibrium points and the corresponding differences between the aggregated impact and $\omega_{1}$ and $\omega_{2}$:
\begin{itemize}
\item point t78: aggimpact-$\omega_{1}$=-0.027005337 $\implies$ accuracy=0.25 and aggimpact-$\omega_{2}$=-0.001819542 $\implies$ accuracy=0.11 
\item point t116: aggimpact-$\omega_{1}$=-0.03167113 6 $\implies$ accuracy=0.25 and aggimpact-$\omega_{2}$=-0.000811289 $\implies$ accuracy=0.11
\end{itemize}

\subsection{Question 2}

The values chosen for the parameters are: repetition=20, duration=10, $\eta_{1}$=0.4, $\eta_{2}$=0.3, $\mu_{1}$=0.8 and $\mu_{2}$=0.9.

Examples of stationary points. \\
\textbf{rep}: t0-t8 \\
\textbf{prep}: t159 	\\
\textbf{feel}: t160

Example of equilibrium point and the corresponding differences between the aggregated impact and $\omega_{1}$ and $\omega_{2}$:
\begin{itemize}
\item point t218: aggimpact-$\omega_{1}$=-0.01893416 $\implies$ accuracy=0.25 and aggimpact-$\omega_{2}$=-1.59672E-05 $\implies$ accuracy=-1.59672E-05/0.000143704
\end{itemize} 


\end{document}
