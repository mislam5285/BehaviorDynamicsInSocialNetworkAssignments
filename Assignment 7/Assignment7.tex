\documentclass[a4paper]{article}

\usepackage[english]{babel}
\usepackage[utf8]{inputenc}
\usepackage{amsmath}
\usepackage{graphicx}
\usepackage[colorinlistoftodos]{todonotes}
\usepackage[export]{adjustbox}[2011/08/13]
\usepackage{float}
\usepackage{bm}

\title{Behaviour Dynamics in Social Networks - Assignment 7}

\author{Maria Hotoiu, Federico Tavella}

\date{\today}

\begin{document}
\maketitle

\begin{abstract}
Verification by mathematical analysis of stationary points.
\end{abstract}

\section{Part A}

The values chosen for the parameters are: $repetition = 30$, $duration = 30$, $\eta_{1} = 0.4$, $\eta_{2} = 0.3$, $\mu_{1} = 0.8$ and $\mu_{2} = 0.9$.

\subsection{Question 1}
The final equilibrium values for the states rep, prep and feel based on observations in the simulation are:

$$rep: 1$$
$$prep: 0.991535551$$
$$feel: 0.89068188.$$

\subsection{Question 2}

The observed equilibrium values based on observation in the simulation:
$$\omega_{1} = 0.832149364$$
$$\omega_{2} = 0.898285371.$$

\noindent The predicted equilibrium values based on mathematical analysis:
$$\omega_{1} = 0.832149364$$
$$\omega_{2} = 0.898285371.$$

The two sets of values for the connection weights are equal, therefore the accuracy is 100\%.

\subsection{Question 3}

The differences between aggregated impact and values for the two adaptive connections for these equilibria vary from 0 to 0.2 in the first case (aggimpact-$\omega_{1}$) and from 0 to 0.1 in the second case (aggimpact-$\omega_{2}$). 

\subsection{Question 4}

todo

\section{Part B}

\subsection{Question 1}

Stationary point from the simulation:


Differences between aggregated impact and values for $\omega_{1}$ and $\omega_{2}$:


\subsection{Question 2}

Now the values of duration and repetition are ???. Stationary point from the simulation:


Differences between aggregated impact and values for $\omega_{1}$ and $\omega_{2}$:

\end{document}